\section{Recurrence in Ergodic Theory}

Let \((X, \mathcal{B}, \mu)\) be a measure space.
We always assume that \(\mu\) is \(\sigma\)-finite, i.e., \(X = \bigcup_{i=1}^{\infty} X_i\) where each \(X_i\) is measurable and \(\mu(X_i) < \infty\).
By ``function on X'' I mean a map \(X \to \bbC\).

\subsection{Measurable transformations}

    \begin{definition}\label{def:measurable_transformation}
        Let \((X,\calB_X,\mu_X)\) and \((Y,\calB_Y,\mu_Y)\) be two measure spaces.
        A map \(T: X \to Y\) is called a \emph{measurable transformation} if for every measurable set \(B \in \calB_Y\), the preimage \(T^{-1}(B) \in \calB_X\) is measurable.

        A measurable transformation \(T: X \to Y\) is called \emph{measure-preserving} if for every measurable set \(B \in \calB_Y\), we have \(\mu_X(T^{-1}(B)) = \mu_Y(B)\).

        A measurable transformation \(T: X \to Y\) is called \emph{invertible} if there exists a measurable transformation \(S: Y \to X\) such that \(S \circ T = \id_X\) and \(T \circ S = \id_Y\).
    \end{definition}

    Note that an invertible measurable transformation is measure-preserving if and only if its inverse is measure-preserving.

    \begin{example}\label{eg:invertible_measurable_transformation_not_measure_preserving}
        Let \(X = \bbR\) with the Borel \(\sigma\)-algebra and the Lebesgue measure.
        Define \(T: X \to X\) by \(T(x) = 2x\).
        Then \(T\) is an invertible measurable transformation, but it is not measure-preserving.
    \end{example}

    \begin{example}\label{eg:measure_preserving_but_not_invertible}
        Let \(X = S^1\) be the unit circle with the Borel \(\sigma\)-algebra and the Lebesgue measure.
        Define \(T: X \to X\) by \(T(z) = z^2\).
        Then \(T\) is a measure-preserving transformation but not invertible.
        
        Indeed, for any connected subset \(A \subseteq S^1\) with arc length \(\theta\), the preimage \(T^{-1}(A)\) consists of two connected subsets, each with arc length \(\theta/2\).
    \end{example}

    \begin{example}\label{eg:measurable_transformation_on_R2}
        Let \(T:\bbR^2 \to \bbR^2\) be defined by \(T(x,y) = (2x, y/2)\) and endow \(\bbR^2\) with Lebesgue measure and the Borel \(\sigma\)-algebra.
        Then \(T\) is an invertible measure-preserving transformation since it preserves the area of rectangles.
    \end{example}

\subsection{Poincar\'e Recurrence Theorem}

    In this subsection, we focus on \(T: X \to X\) where \((X, \calB, \mu)\) is a measure space.

    \begin{definition}\label{def:recurrent}
        Let \((X, \mathcal{B}, \mu)\) be a measure space and \(T: X \to X\) be a measure transformation.
        Let \(A \in \mathcal{B}\) be a measurable set.
        A point \(x \in A\) is called \emph{recurrent} if \(T^n(x) \in A\) for some integer \(n > 0\).
    \end{definition}

    \begin{theorem}[Poincar\'e Recurrence Theorem]\label{thm:poincare_recurrence_theorem}
        Let $(X, \mathcal{B}, \mu)$ be a measure space with finite measure $\mu(X) < \infty$, 
        and let $T: X \to X$ be a measure-preserving transformation, i.e., for all $A \in \mathcal{B}$, $\mu(T^{-1}(A)) = \mu(A)$. 
        Then for any measurable set $A$ with $\mu(A) > 0$, the for almost every point \(x \in A\), there exist infinitely many integers \(n > 0\) such that \(T^n(x) \in A\).
    \end{theorem}
    \begin{proof}
        \Yang{To be continued.}
    \end{proof}

    Let \(f\) be the characteristic function of \(A\), then Poincar\'e Recurrence Theorem states that for almost every \(x \in A\), the sequence \(\sum_{n=1}^{\infty} f(T^n(x))\) diverges.

    Hence we have a generalization.

    \begin{theorem}\label{thm:generalized_poincare_recurrence_theorem}
        Let \(T:X \to X\) be a measure-preserving transformation on a measure space \((X, \calB, \mu)\) with \(\mu(X) < \infty\). 
        Let \(f: X \to [0, \infty)\) be a non-negative measurable function.
        Then for almost every point \(x \in \{y| f(y) > 0\}\), the sequence \(\sum_{n=1}^{\infty} f(T^n(x))\) diverges.
        \Yang{To be checked}
    \end{theorem}
    \begin{proof}
        \Yang{To be completed.}
    \end{proof}



\subsection{Mean Convergence}

    Let \((X, \mathcal{B}, \mu)\) be a measure space with finite measure \(\mu(X) < \infty\), and let \(T: X \to X\) be a measure transformation.
    Let \(E \in \calB\) be a measurable set.

    Given a point \(x \in X\) and given \(N \in \bbN\), for the ratio \(\eta\) the number of these points \(\{n \in 1,...,N: T^n(x) \in A\}\) to \(N+1\), we want to find the limit of \(\eta\) as \(N \to \infty\).
    \Yang{To be checked}

    Let \(f\) denote the characteristic function of \(E\), i.e.,
    \[
        f(x) = \begin{cases}
            1, & x \in E, \\
            0, & x \notin E.
        \end{cases}
    \]
    Then the ratio \(\eta_N\) can be expressed as
    \[
        \eta_N = \frac{1}{N+1} \sum_{n=0}^{N} f(T^n(x)).
    \]
    We are interested in the limit
    \[   \lim_{N \to \infty} \eta_N = \lim_{N \to \infty} \frac{1}{N+1} \sum_{n=0}^{N} f(T^n(x)). \]

    \Yang{By ``function on X'' I mean a map \(X \to \bbC\). To be checked}

    Let \(f\) be a measurable function on \(X\) and \(T: X \to X\) be a measure transformation.
    Let \(g:X \to \bbC\) be defined by \(g(x) = f(T(x))\).
    We usually write \(g = U_f\).
    More precisely, for every function \(f\) on \(X\), we define \(U_f = f \circ T\).
    This gives a linear operator \(U: f \mapsto U_f\) on the space of measurable functions on \(X\).

    \begin{proposition}\label{prop:U_isometry_on_L1}
        Let \((X, \mathcal{B}, \mu)\) be a measure space with finite measure \(\mu(X) < \infty\), and let \(T: X \to X\) be a measure-preserving transformation.
        Let \
        \[ L^1(X, \mu) = \{ f: X \to \bbC \mid f \text{ is measurable and } \int_X |f| d\mu < \infty \}. \]
        Then \(U\) maps \(L^1(X, \mu)\) to itself and is an isometry, 
        i.e., for every \(f \in L^1(X, \mu)\), we have \(U_f \in L^1(X, \mu)\) and \(\|U_f\|_1 = \|f\|_1\).
    \end{proposition}
    \begin{proof}
        \Yang{To be completed}
    \end{proof}
    
    More generally, for \(1 \leq p \leq \infty\), let
    \[ L^p(X, \mu) = \{ f: X \to \bbC \mid f \text{ is measurable and } \|f\|_p < \infty \}, \]
    where
    \[
        \|f\|_p = \begin{cases}
            \left( \int_X |f|^p d\mu \right)^{1/p}, & 1 \leq p < \infty, \\
            \text{ess sup}_{x \in X} |f(x)|, & p = \infty.
        \end{cases}
    \]
    Then \(L^p(X, \mu)\) is a Banach space with the norm \(\|\cdot\|_p\).
    \cref{prop:U_isometry_on_L1} can be generalized to \(L^p\).

    \paragraph{Special case: \(p=2\)}. On \(L^2(X, \mu)\), we can define an inner product by
    \[ \langle f, g \rangle = \int_X f \overline{g} d\mu. \]
    Then \(L^2(X, \mu)\) is a Hilbert space with the inner product \(\langle \cdot, \cdot \rangle\).

    \begin{definition}\label{def:unitary_operator}
        Let \(\calH\) be a Hilbert space.
        A linear operator \(U: \calH \to \calH\) is called an \emph{isometry} if for every \(x \in \calH\), \(\|Ux\| = \|x\|\).
        The operator \(U\) is called \emph{unitary} if \(<Ux, Uy> = <x, y>\) for every \(x, y \in \calH\).
    \end{definition}

    \begin{proposition}\label{prop:unitary_iff_isometry_and_surjective}
        Let \(U: \calH \to \calH\) be a linear operator on a Hilbert space \(\calH\).
        Then \(U\) is isometry if and only if \(U^* U = \id_\calH\), where \(U^*\) is the adjoint operator of \(U\).
        Moreover, \(U\) is unitary if and only if \(U^* U = U U^* = \id_\calH\).
        In particular, \(U\) is unitary if and only if \(U\) is an isometry and surjective.
    \end{proposition}

    \begin{proposition}\label{prop:U_is_unitary_on_L2}
        Let \((X, \mathcal{B}, \mu)\) be a measure space with finite measure \(\mu(X) < \infty\), and let \(T: X \to X\) be a measure-preserving transformation which is also invertible.
        Then the operator \(U: L^2(X, \mu) \to L^2(X, \mu)\) defined by \(U_f = f \circ T\) is unitary.
    \end{proposition}
    \begin{proof}
        \Yang{To be completed}
    \end{proof}

    Recall the limit we want to find:
    \[ \lim_{N \to \infty} \frac{1}{N+1} \sum_{n=0}^{N} f(T^n(x)). \]
    This can be rewritten as
    \[ \lim_{N \to \infty} \frac{1}{N+1} \sum_{n=0}^{N} U^n f(x). \]
    So let us focus on the limit of
    \[ \frac{1}{N+1} \sum_{n=0}^{N} U^n. \]

    In the case \(\calH = \bbC\), the unitary operator \(U: \bbC \to \bbC\) is just a multiplication by a complex number \(e^{i\theta}\) with \(\theta \in \bbR\).
    Then the limit
    \[ \lim_{N \to \infty} \frac{1}{N+1} \sum_{n=0}^{N} U^n \]
    converges for all \(\theta\) and the limit is given by
    \[ \lim_{N \to \infty} \frac{1}{N+1} \sum_{n=0}^{N} U^n = \begin{cases}
        1, & \exp(i\theta) = 1, \\
        0, & \text{otherwise}.
    \end{cases} \]

    In the case \(\calH = \bbC^n\), the unitary operator \(U: \bbC^n \to \bbC^n\) can be represented by a unitary matrix.
    Then the limit
    \[ \lim_{N \to \infty} \frac{1}{N+1} \sum_{n=0}^{N} U^n = P\]
    is the orthogonal projection onto the subspace of \(\bbC^n\) spanned by the eigenvectors of \(U\) corresponding to the eigenvalue \(1\).

    \begin{theorem}[von Neumann Mean Ergodic Theorem]\label{thm:von_neumann_mean_ergodic_theorem}
        Let \(\calH\) be a Hilbert space and \(U: \calH \to \calH\) be an isometry and \(P\) the orthogonal projection onto the subspace of \(\calH\) consisting of all vectors \(y\) such that \(U y = y\).
        Then the limit
        \[ \lim_{N \to \infty} \frac{1}{N+1} \sum_{n=0}^{N} U^n\]
        converges to \(P\).
        \Yang{To be checked}
        \Yang{What is the projection exactly?}
    \end{theorem}
    \begin{proof}
        \Yang{To be completed}
        \Yang{Hint: \(\calH = \ker(U - I) \oplus \overline{\Image(U - I)}\) and this is the orthogonal decomposition.}
    \end{proof}

    \begin{lemma}\label{lem:U_xi_equals_xi_iff_U_star_xi_equals_xi}
        If \(U: \calH \to \calH\) is an isometry, then \(U \xi = \xi\) if and only if \(U^* \xi = \xi\).
    \end{lemma}
    \begin{proof}
        \Yang{To be completed}
    \end{proof}

    \begin{theorem}[Birkhoff Ergodic Theorem]\label{thm:Birkhoff_Ergodic_Theorem}
        Let \(T:X\to X\) be a measure-preserving transformation on a measure space \((X, \mathcal{B}, \mu)\). 
        Let \(f\in L^1(X, \mu)\). 
        Then 
        \[ \frac{1}{n} \sum_{i=0}^{n-1} f(T^i x) \]
        converges for almost every \(x\in X\).
        The limit function \(\tilde{f}\) is integrable and \(T\)-invariant, i.e., \(\tilde{f}(Tx) = \tilde{f}(x)\) for almost every \(x\in X\).
        If \(\mu(X) < \infty\), then
        \[ \int_X f \upd\mu = \int_X \tilde{f} \upd\mu. \]
    \end{theorem}
    \begin{proof}
        We may assume that \(f\) is real-valued, since we can apply the theorem to the real and imaginary parts of \(f\) separately.

        \Yang{To be continued...}
    \end{proof}

    \begin{theorem}[Maximal Ergodic Theorem]\label{thm:Maximal_Ergodic_Theorem}
        Let \(U: L_\bbR^1 \to L_\bbR^1\) be a linear operator such that
        \begin{enumerate}
            \item \(\|Uf\|_1 \leq \|f\|_1\) for all \(f\in L^1\);
            \item \(Uf \geq_{a.e.} 0\) if \(f\geq_{a.e.} 0\).
        \end{enumerate}
        Let \(N>0\) and \(f \in L^1_\bbR(X, \mu)\). 
        Define \(f_0 = 0\) and \(f_n = Uf_{n-1} + f\) for \(n\geq 1\).
        Let \(F_N = \max_{0\leq n \leq N} f_n\) and \(E = \{x\in X: F_N(x) > 0\}\).
        Then 
        \[
            \int_E f \upd\mu \geq 0.
        \]
    \end{theorem}
    \begin{proof}
        \Yang{To be continued...}
    \end{proof}