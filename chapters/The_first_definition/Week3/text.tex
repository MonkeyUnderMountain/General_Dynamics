\section{Week 3}


\subsection{}

    \begin{theorem}[Birkhoff Ergodic Theorem]\label{thm:Birkhoff_Ergodic_Theorem}
        Let \(T:X\to X\) be a measure-preserving transformation on a measure space \((X, \mathcal{B}, \mu)\). 
        Let \(f\in L^1(X, \mu)\). 
        Then 
        \[ \frac{1}{n} \sum_{i=0}^{n-1} f(T^i x) \]
        converges for almost every \(x\in X\).
        The limit function \(\tilde{f}\) is integrable and \(T\)-invariant, i.e., \(\tilde{f}(Tx) = \tilde{f}(x)\) for almost every \(x\in X\).
        If \(\mu(X) < \infty\), then
        \[ \int_X f \upd\mu = \int_X \tilde{f} \upd\mu. \]
    \end{theorem}
    \begin{proof}
        We may assume that \(f\) is real-valued, since we can apply the theorem to the real and imaginary parts of \(f\) separately.

        \Yang{To be continued...}
    \end{proof}

    \begin{theorem}[Maximal Ergodic Theorem]\label{thm:Maximal_Ergodic_Theorem}
        Let \(U: L_\bbR^1 \to L_\bbR^1\) be a linear operator such that
        \begin{enumerate}
            \item \(\|Uf\|_1 \leq \|f\|_1\) for all \(f\in L^1\);
            \item \(Uf \geq_{a.e.} 0\) if \(f\geq_{a.e.} 0\).
        \end{enumerate}
        Let \(N>0\) and \(f \in L^1_\bbR(X, \mu)\). 
        Define \(f_0 = 0\) and \(f_n = Uf_{n-1} + f\) for \(n\geq 1\).
        Let \(F_N = \max_{0\leq n \leq N} f_n\) and \(E = \{x\in X: F_N(x) > 0\}\).
        Then 
        \[
            \int_E f \upd\mu \geq 0.
        \]
    \end{theorem}
    \begin{proof}
        \Yang{To be continued...}
    \end{proof}


\subsection{Ergodic}

    \begin{definition}\label{def:ergodic_transformation}
        A measure-preserving transformation \(T:X\to X\) on a measure space \((X, \mathcal{B}, \mu)\) is said to be \emph{ergodic} if it has no nontrivial invariant sets, i.e., for every \(A\in \mathcal{B}\) such that \(T^{-1}A = A\), we have \(\mu(A) = 0\) or \(\mu(X\setminus A) = 0\).
        
        % \Yang{}for every \(A\in \mathcal{B}\) such that \(T^{-1}A = A\), we have \(\mu(A) = 0\) or \(\mu(X\setminus A) = 0\).
    \end{definition}

    \begin{theorem}\label{thm:characterization_of_ergodic_transformation}
        Let \((X, \mathcal{B}, \mu)\) be a probability space and \(T:X\to X\) be a measure-preserving transformation. 
        Then the following statements are equivalent:
        \begin{enumerate}
            \item \(T\) is ergodic;
            \item for each measurable set \(A\), \(m((T^{-1}A) \Delta A) = 0\) iff \(\mu(A) = 0\) or \(\mu(X\setminus A) = 0\);
            \item for each measurable set \(A\) with \(\mu(A) > 0\), we have \(\mu\left(\bigcup_{n=0}^\infty T^{-n}A\right) = 1\);
            \item for each measurable sets \(A, B\) with \(\mu(A)\mu(B) > 0\), there exists \(n\in \bbZ^+\) such that \(\mu(T^{-n}A \cap B) > 0\).
        \end{enumerate}
    \end{theorem}

    % \begin{proposition}\label{prop:ergodic_imply_invariant_function_constant_ae}
    %     Let \((X, \mathcal{B}, \mu)\) be a probability space and \(T:X\to X\) be a measure-preserving transformation.
    %     Then \(T\) is ergodic iff any invariant function \(f\in L^1(X, \mu)\) is constant almost everywhere.
    % \end{proposition}

    \begin{theorem}\label{thm:ergodic_imply_invariant_function_constant_ae}
        Let \((X, \mathcal{B}, \mu)\) be a measure space with \(\mu(X) < \infty\) and \(T:X\to X\) be a measure-preserving transformation.
        Then TFAE:
        \begin{enumerate}
            \item \(T\) is ergodic;
            \item for each measurable function \(f\) which is \(T\)-invariant, \(f\) is constant almost everywhere;
            \item for each \(f\in L^1(X, \mu)\) which is \(T\)-invariant, \(f\) is constant almost everywhere;
            \item for each \(f\in L^2(X, \mu)\) which is \(T\)-invariant, \(f\) is constant almost everywhere.
        \end{enumerate}
    \end{theorem}
    \begin{proof}
        \Yang{To be continued...}

        We show that (a) \(\Rightarrow\) (b).
    \end{proof}
 
    \begin{example}\label{eg:transition_of_natural_numbers}
        Let \(X = \bbN\) and \(\mathcal{B} = \mathcal{P}(\bbN)\) with the counting measure.
        Define \(T:\bbN \to \bbN\) by \(Tx = x + 1\).
        Then \(T\) is measure-preserving and ergodic.
    \end{example}

    \begin{example}\label{eg:transition_by_2_of_natural_numbers}
        Let \(X = \bbN\) and \(\mathcal{B} = \mathcal{P}(\bbN)\) with the counting measure.
        Define \(T:\bbN \to \bbN\) by \(Tx = x + 2\).
        Then \(T\) is measure-preserving but not ergodic.
    \end{example}

    \begin{example}\label{eg:transition_by_1_of_real_numbers}
        Let \(X = \bbR\) and \(\mathcal{B} = \mathcal{B}(\bbR)\) with the Lebesgue measure.
        Define \(T:\bbR \to \bbR\) by \(Tx = x + 1\).
        Then \(T\) is measure-preserving but not ergodic.
    \end{example}

    \begin{example}\label{eg:rotation_on_unit_circle}
        Let \(X = S^1 = \{z\in \bbC: |z| = 1\}\) and \(\mathcal{B} = \mathcal{B}(S^1)\) with the Lebesgue measure.
        Define \(T:S^1 \to S^1\) by \(Tx = \upe^{2\pi i \theta} x\) where \(\theta \in \bbR\).
        Then \(T\) is measure-preserving.
        Moreover, \(T\) is ergodic iff \(\theta\) is irrational.
    \end{example}

    \begin{proposition}\label{prop:irrational_rotation_is_ergodic}
        In Example \ref{eg:rotation_on_unit_circle}, if \(\theta\) is irrational, then \(T\) is ergodic.
    \end{proposition}
    \begin{proof}
        \Yang{To be continued..., By Fourier series}
    \end{proof}

    \begin{example}
        Let \(X = S^1\) and \(\mathcal{B} = \mathcal{B}(S^1)\) with the Lebesgue measure.
        Define \(T:S^1 \to S^1\) by \(Tx = x^2\).
        Then \(T\) is measure-preserving and ergodic.
    \end{example}